\section[JSP Sess�es]{JSP Sess�es}

\begin{frame}[c,allowframebreaks,fragile]
    \frametitle{JSP Sess�es}
    \begin{itemize}
      \item em JSP as sess�es (assim como o \textit{request} e
\textit{response}) s�o pr�-inicializados internamente;
      \item n�o precisamos pegar a sess�o armazenada no \textit{request} para
manipular as vari�veis de sess�o;
    \end{itemize}
    
    \begin{center}
    \begin{minipage}{0.9\textwidth}
	\lstset{
	  frame=tb,
	  language=Java,
	  caption=Setando as sess�es
	}
	\begin{lstlisting}
	  HttpSession session = request.getSession();
	  session.setAttribute("usuario_autenticado", usuario.getNome());
	  session.setAttribute("id_usuario_autenticado", usuario.getId());
	\end{lstlisting}
    \end{minipage}
    \end{center}
    
    \begin{itemize}
      \item usando apenas JSP a primeira linha n�o � necess�ria;
      
      \framebreak
      
      \item o m�todo \texttt{setAttribute} inclui um valor e um identificador na
sess�o;
      \item para verificar ou receber esse valor armazenado usamos
o \textit{getAttribute}; 
    \end{itemize}
    
    \begin{center}
    \begin{minipage}{0.9\textwidth}
	\lstset{
	  frame=tb,
	  language=Java,
	  caption=Recuperando as sess�es
	}
	\begin{lstlisting}
	  String usuarioAutenticado = (String)
	  session.getAttribute("usuario_autenticado");
	  Long idUsuarioAutenticado = (Long)
	  session.getAttribute("id_usuario_autenticado");
	  if (usuarioAutenticado == null || idUsuarioAutenticado == null) {
	    response.sendRedirect("login.jsp");
	  }
	\end{lstlisting}
    \end{minipage}
    \end{center}
    
    \framebreak
    
    \begin{itemize}
      \item para destruir a sess�o removendo todos os valores armazenados
usamos o c�digo abaixo, isso � �til quando um usu�rio faz \textit{logout} no
sistema.
    \end{itemize}
    
    \begin{center}
    \begin{minipage}{0.9\textwidth}
	\lstset{
	  frame=tb,
	  language=Java,
	  caption=Destruindo as sess�es
	}
	\begin{lstlisting}
	  session.invalidade();
	\end{lstlisting}
    \end{minipage}
    \end{center}
    
    \framebreak
    
    \begin{itemize}
      \item pode-se remover um valor espec�fico da sess�o usando o c�digo de
exemplo abaixo.
    \end{itemize}
    
    \begin{center}
    \begin{minipage}{0.9\textwidth}
	\lstset{
	  frame=tb,
	  language=Java,
	  caption=Removendo uma �nica sess�o
	}
	\begin{lstlisting}
	  HttpSession session = request.getSession();
	  session.removeAttribute("usuario_autenticado");
	\end{lstlisting}
    \end{minipage}
    \end{center}
\end{frame}

\begin{frame}
    \frametitle{Atividades}
      \begin{itemize}
       \item Fa�a um sistema de Login, contendo:
       \begin{itemize}
	  \item E-mail do usu�rio (validado por javascript);
	  \item Senha so usu�rio;
	  \item Acesso permitido caso o email seja ``admin@utfpr.edu.br'' e a
senha seja ``posJava2012'';
	  \item Mensagem de acesso permitido ou negado;
	  \item Sistema de LogOff (removendo as sess�es setadas);
       \end{itemize}
      \end{itemize}
\end{frame}
