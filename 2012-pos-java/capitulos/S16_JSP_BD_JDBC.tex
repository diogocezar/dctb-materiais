\section[JSP Banco de Dados (JDBC)]{JSP Banco de Dados (JDBC)}

\begin{frame}[c,allowframebreaks,fragile]
    \frametitle{JSP Banco de Dados (JDBC)}
    \begin{itemize}
      \item \textit{Java Database Connectivity} ou JDBC � um conjunto de classes
e interfaces (API) escritas em Java que fazem o envio de instru��es SQL para
qualquer banco de dados relacional;
      \item Api de baixo n�vel e base para api's de alto n�vel;
      \item Amplia o que voc� pode fazer com Java;
      \item Possibilita o uso de bancos de dados j� instalados; 
      \item Para cada banco de dados h� um driver JDBC;
      
      \framebreak
      
      \item exemplo separado em arquivos:
      
      \begin{itemize}
	\item bd:
	\begin{itemize}
	 \item ConnectionBase - � a base para a conex�o ao banco de dados;
	 \item DataBase - Herda a ConnectionBase e configura o banco de dados
baseado nas constantes definidas na classe ConfigSite;
	\end{itemize}
	\item config:
	\begin{itemize}
	  \item ConfigSite - Seta as configura��es do banco e do site em
quest�o;
	\end{itemize}
	\item errors:
	\begin{itemize}
	  \item Error - Seta as mensagens de erro do sistema;
	\end{itemize}
	\item util:
	\begin{itemize}
	  \item Library - Biblioteca com fun��es do sistema;
	\end{itemize}
      \end{itemize}
    \end{itemize}
\end{frame}

\begin{frame}[c,allowframebreaks,fragile]
    \frametitle{Exemplos}
    
    \begin{center}
    \begin{minipage}{0.9\textwidth}
	\lstset{
	  frame=tb,
	  language=Java,
	  caption=Exemplo ConnectionBase
	}
	\begin{lstlisting}
public ConnectionBase(String url, String bd, String user, String key,
String path, String driver, int tipo, int concorrencia){
	...
	try {
		  Class.forName(getDriver());
		  setConnection(DriverManager.getConnection(getUrl(), getUser(),
	  getKey()));
		  this.connection.setAutoCommit(isAutoCommit());
		  setStatement(this.connection.createStatement(getTipo(),
	  getConcorrencia()));
		  if(!Library.empty(getPath())){
		      this.statement.execute("SET SEARCH_PATH TO " + getPath());
		  }
		  setConectado(true);
	      } catch (Exception ex) {
		  Error.addError(Error.E_CONNECTION + ex);
		  setConectado(false);
	      }
	  }
	\end{lstlisting}
    \end{minipage}
    \end{center}
    
    \framebreak
    
    \begin{center}
    \begin{minipage}{0.9\textwidth}
	\lstset{
	  frame=tb,
	  language=Java,
	  caption=Exemplo DataBase
	}
	\begin{lstlisting}
	  public class DataBase extends ConnectionBase{
	    public DataBase(){        
	      super(ConfigSite.url, ConfigSite.bd,
ConfigSite.user, ConfigSite.key, ConfigSite.path,
ConfigSite.driver, ConfigSite.tipo, ConfigSite.concorrencia);
	    }
	    public ResultSet query(String sql){}
	    public int write(String sql){}
	    public void close(){}
	    public int insert(String tabela, String campos[], String
valores[]){}
	    public int update(String tabela, String condicao, String campos[],
	    String valores[]){}
	    public int delete(String tabela, String condicao){}
	    public boolean conectado(){}
	    public void commit(){}
	  }
	\end{lstlisting}
    \end{minipage}
    \end{center}
    
    \framebreak
    
    \begin{center}
    \begin{minipage}{0.9\textwidth}
	\lstset{
	  frame=tb,
	  language=Java,
	  caption=Exemplo Error
	}
	\begin{lstlisting}
	public abstract class ConfigSite {
	  public static final String TITULO_SISTEMA = "Alunos Online";
	  public static String url = "jdbc:postgresql://localhost:5432/";
	  public static String bd = "alunos";
	  public static String user = "postgres";
	  public static String key = "*****";
	  public static String path = "public";
	  public static String driver = "org.postgresql.Driver";
	  public static boolean autoCommit = false;
	  public static int tipo = ResultSet.TYPE_SCROLL_SENSITIVE;
	  public static int concorrencia = ResultSet.CONCUR_UPDATABLE;
	  private ConfigSite(){
	  }    
	}
	\end{lstlisting}
    \end{minipage}
    \end{center}
    
    \framebreak
    
    \begin{center}
    \begin{minipage}{0.9\textwidth}
	\lstset{
	  frame=tb,
	  language=Java,
	  caption=Exemplo ConfigSite
	}
	\begin{lstlisting}
	  public abstract class Error {
	    public static final String E_CONNECTION = "Erro ao se conectar com
o banco de dados: ";
	    public static final String E_QUERY = "Erro ao executar uma
consulta no banco de dados: ";
	    public static final String E_WRITE = "Erro ao executar uma escrita
no banco de dados: ";
	    public static final String E_CLOSE = "Erro ao fechar a conex�o com
o banco de dados: ";
	    public static final String E_COMMIT = "Erro ao efetuar o commit: ";
	    private static String errors = "";   
	    public static void addError(String error){
	    setErrors(errors+"<br>"+error);
	    }
	    public static void clearError(){
	    setErrors("");
	    }
	    ...
	  }
	\end{lstlisting}
    \end{minipage}
    \end{center}
    
    \framebreak
    
    \begin{center}
    \begin{minipage}{0.9\textwidth}
	\lstset{
	  frame=tb,
	  language=Java,
	  caption=Exemplo Library
	}
	\begin{lstlisting}
	  package util;
	  public abstract class Library {
	      public static boolean empty(String vazia){
		  if(vazia.compareTo("") == 0){
		      return true;
		  }
		  else{
		      return false;
		  }
	      }
	      private Library() {
	      }
	  }
	\end{lstlisting}
    \end{minipage}
    \end{center}
    
\end{frame}


\begin{frame}
    \frametitle{Atividades}
      \begin{itemize}
       \item Utilize esta estrutura de arquivos para fazer um crud do objeto
\textit{Produto}:
       \begin{itemize}
	  \item id (int);
	  \item Nome (string);
	  \item Categoria (string);
	  \item Pre�o (Float);
	  \item Promo��o (Boolean);
	  \item Desconto (Float);
       \end{itemize}
      \end{itemize}
\end{frame}
