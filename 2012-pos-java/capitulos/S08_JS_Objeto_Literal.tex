\section[JavaScript Objeto Literal]{JavaScript Objeto Literal}

\begin{frame}[c,allowframebreaks,fragile]
    \frametitle{JavaScript Objeto Literal}
    \begin{itemize}
      \item objeto literal � uma forma de organizar o c�digo
\textit{JavaScript};

      \item recomenda-se: cada arquivo que necessite usar qualquer tipo de
intera��o \textit{JavaScript} deve conter um objeto literal correspondente;

      \item cria��o de um Objeto Literal:

    
    \begin{center}
    \begin{minipage}{0.9\textwidth}
	\lstset{
	  frame=tb,
	  language=HTML,
	  caption=Exemplo Objeto Literal
	}
	\begin{lstlisting}
	  var ObjetoLiteral = {

	  }
	\end{lstlisting}
    \end{minipage}
    \end{center}
    
    \framebreak
    
    \item elementos n�o possuem distin��o;
    
    \item podem ser:
    
    \begin{itemize}
     \item atributos;
     \item fun��es;
     \item outros objetos;
    \end{itemize}

    \item exemplo de declara��o destes elementos:
    
    \begin{center}
    \begin{minipage}{0.9\textwidth}
	\lstset{
	  frame=tb,
	  language=HTML,
	  caption=Exemplo Objeto Literal - Elementos
	}
	\begin{lstlisting}
	  var ObjetoLiteral = {
	    contador: 0,
	    init: function(){}
	  }
	\end{lstlisting}
    \end{minipage}
    \end{center}
    
    \item atribui��o feita por ``:'';
    
    \item separa��o entre os elementos por ``,'';
    
    \framebreak
    
    \item conven��es jQuery e Objeto Literal:
    
    \begin{itemize}
     \item criar uma fun��o de inicializa��o que ir� ser executada ao carregar a
p�gina;
     \item essa fun��o � respons�vel por linkar os componentes HTML com as
fun��es do Objeto Literal;
    \end{itemize}
    
    \begin{center}
    \begin{minipage}{0.9\textwidth}
	\lstset{
	  frame=tb,
	  language=HTML,
	  caption=Exemplo Objeto Literal - Init
	}
	\begin{lstlisting}
	  var ObjetoLiteral = {
	  init: function(){
	    $("#id_div").click(ObjetoLiteral.acao);
	  },
	    acao: function(){ alert("Testando um ObjetoLiteral"); }
	  }
	\end{lstlisting}
    \end{minipage}
    \end{center}
    
    \framebreak
    
    \item essa fun��o de inicializa��o (No exemplo init) dever� ser invocada no
m�todo \textit{ready} do \textit{jQuery}:

    \begin{center}
    \begin{minipage}{0.9\textwidth}
	\lstset{
	  frame=tb,
	  language=HTML,
	  caption=Exemplo Objeto Literal - Chamda Init
	}
	\begin{lstlisting}
	  $(document).ready(function(){
	    ObjetoLiteral.init();
	  });
	\end{lstlisting}
    \end{minipage}
    \end{center}

    \framebreak
    
    \item Um exemplo completo:
    
    \begin{center}
    \begin{minipage}{0.9\textwidth}
	\lstset{
	  frame=tb,
	  language=HTML,
	  caption=Exemplo Completo
	}
	\begin{lstlisting}
	  var ObjetoLiteral = {
	    init: function(){
	      $("#id_div").click(ObjetoLiteral.acao);
	    },
	    acao: function(){ alert("Testando um ObjetoLiteral"); }
	  }
	  $(document).ready(function(){
	    ObjetoLiteral.init();
	  });
	\end{lstlisting}
    \end{minipage}
    \end{center}
    \end{itemize}
\end{frame}

\begin{frame}
    \frametitle{Atividades}
      \begin{itemize}
       \item utilizar o script \texttt{validate.js} e criar um formul�rio com as
seguintes restri��es:
       \begin{itemize}
	\item nome (Obrigat�rio);
	\item email (Obrigat�rio e Validado);
	\item telefone (Opcional);
	\item endere�o (Obrigat�rio);
	\item CPF (Obrigat�rio).
       \end{itemize}
      \end{itemize}
\end{frame}
