\section[JSP Cookies]{JSP Cookies}

\begin{frame}[c,allowframebreaks,fragile]
    \frametitle{JSP Cookies}
    \begin{itemize}
      \item \textit{Cookie} � um mecanismo padr�o fornecido pelo protocolo HTTP
e que permite gravarmos pequenas quantidades de dados persistentes no navegador
de um usu�rio;
      \item tais dados podem ser recuperados posteriormente pelo navegador;
      \item esse mecanismo � usado quando queremos recuperar informa��es de
algum usu�rio;
      \item com os \textit{cookies}, pode-se reconhecer quem entra num site, de
onde vem, com que periodicidade costuma voltar;

      \framebreak

      \item como padr�o, os cookies expiram t�o logo o usu�rio encerra a
navega��o naquele site, por�m podemos configur�-los para persistir por v�rios
dias;
      \item al�m dos dados que ele armazena, um \textit{cookie} recebe um nome;
      \item um servidor pode ent�o definir m�ltiplos cookies e fazer a
identifica��o entre eles atrav�s dos seus nomes;
      \item os cookies s�o associados a URL da p�gina que os manipula.
    \end{itemize}
    
    \framebreak
    
    \begin{center}
    \begin{minipage}{0.9\textwidth}
	\lstset{
	  frame=tb,
	  language=Java,
	  caption=Setando as cookies
	}
	\begin{lstlisting}
	Cookie cookie = new Cookie("txtNome",
request.getParameter("txtNome"));
	cookie.setMaxAge(365 * 24 * 60 * 60);
	response.addCookie(cookie);
	\end{lstlisting}
    \end{minipage}
    \end{center}
    
    \framebreak
    
    \begin{center}
    \begin{minipage}{0.9\textwidth}
	\lstset{
	  frame=tb,
	  language=Java,
	  caption=Recuperando as cookies
	}
	\begin{lstlisting}
	Cookie txtNome = null;
	Cookie mCookies[] = request.getCookies();
	for(int i = 0; i < mCookies.length; i++){
	  if(mCookies[i].getName().equals("txtNome")){
	    txtNome = mCookies[i];
	  }
	}
	...
	<h1> Ol�, <%= txtNome.getValue() %> voc� tamb�m � reconhecido aqui!</h1>
	\end{lstlisting}
    \end{minipage}
    \end{center}
    
    \framebreak
    
    \begin{center}
    \begin{minipage}{0.9\textwidth}
	\lstset{
	  frame=tb,
	  language=Java,
	  caption=Excluindo as cookies
	}
	\begin{lstlisting}
	  txtNome.setMaxAge(1);
	  response.addCookie(txtNome);
	\end{lstlisting}
    \end{minipage}
    \end{center}
    
\end{frame}




\begin{frame}
    \frametitle{Atividades}
      \begin{itemize}
       \item Atualize o sistema de login adicionando as funcionalidades:
       \begin{itemize}
	  \item salvar o email do usu�rio e no pr�ximo acesso recuperar este
email;
	  \item armazenar a data do ultimo login em cookie e exib�-la caso o
acesso seja permitido;
       \end{itemize}
      \end{itemize}
\end{frame}
