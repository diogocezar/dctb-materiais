\section[JSP Banco de Dados (Hibernate)]{JSP Banco de Dados (Hibernate)}

\begin{frame}[c,allowframebreaks,fragile]
    \frametitle{JSP Banco de Dados (Hibernate)}
    \begin{itemize}
      \item o \textit{Hibernate} � um \textit{framework} para o mapeamento
objeto-relacional escrito na linguagem Java;
      \item este \textit{framework} facilita o mapeamento dos atributos entre
uma base tradicional de dados relacionais e o modelo objeto de uma aplica��o;
      \item uso de arquivos (XML) ou anota��es Java;
    
    
    \framebreak 
    
    \item Arquivo de configura��o: \textit{hibernate.cfg.xml}
    
  \begin{center}
  \begin{minipage}{0.9\textwidth}
      \lstset{
	frame=tb,
	language=XML
      }
      \begin{lstlisting}
	<hibernate-configuration>
	<session-factory>
	<property name="hibernate.dialect">org.hibernate.dialect.PostgreSQLDialect</property>
	<propertyname="hibernate.connection.driver_class">org.postgresql.Driver</property>
	<propertyname="hibernate.connection.url">jdbc:postgresql://localhost:5432/alunos</property>
	<property name="hibernate.connection.username">postgres</property>
	<property name="hibernate.connection.password">*****</property>
	<property name="show_sql">true</property>
	<property name="hibernate.hbm2ddl.auto">update</property>
	<mapping class="model.Aluno"/>
	<mapping class="model.Endereco"/>
	</session-factory>
	</hibernate-configuration>
      \end{lstlisting}
  \end{minipage}
  \end{center}
  
  \framebreak
  
  \item MVC: � um modelo de desenvolvimento de \textit{Software};

  \item � um padr�o de arquitetura de aplica��es que visa
separar a l�gica da aplica��o (\textit{Model}), da interface do usu�rio (\textit{View}), e do fluxo da aplica��o (\textit{Controller});
  
  \item \textit{Model} (Modelo): representa os dados;
  
  \item \textit{View} (Vis�o): representa a interface de intera��o com o usu�rio;
  
  \item \textit{Controller} (Controle): representa o elo de liga��o entre vis�o e modelo, � repons�vel tamb�m por tratar a regra de neg�cios;
  
  \framebreak
  
  \item Organiza��o dos arquivos:
  
  \begin{itemize}
    \item ClasseField e ClasseList: Respons�vel por tratar a itera��o com o usu�rio;
    \item ClasseDAO: Respons�vel por invocar as intera��es com os dados de um modelo;
    \item Classe: Modelo que representa os campos da tabela;
  \end{itemize}
  
  \framebreak
  
  \item Exemplo Aluno - AlunoDAO:
  
  \begin{center}
  \begin{minipage}{0.9\textwidth}
      \lstset{
	frame=tb,
	language=Java
      }
      \begin{lstlisting}
	public class AlunoDAO {
	  private SessionFactory factory;
	  public AlunoDAO(){
	    ...
	  }
	  public List<Aluno> getAllList(){
	    ...
	  }
	  public void insert(Aluno aluno){
	    ...
	  }
	}
      \end{lstlisting}
  \end{minipage}
  \end{center}
  
  \framebreak
  
  \item Exemplo aluno - Aluno:
  
  \begin{center}
  \begin{minipage}{0.9\textwidth}
      \lstset{
	frame=tb,
	language=Java
      }
      \begin{lstlisting}
	@Table(name="aluno")
	@Entity
	public class Aluno {

	@Id
	@SequenceGenerator(name = "idSeqAluno", sequenceName="aluno_id_seq")
	@GeneratedValue(generator = "idSeqAluno")
	private Integer id;

	@Column(name = "nome")
	private String nome;

	@Column(name = "idade")
	private Integer idade;

	@Column(name = "fone")
	private String fone;

	@Column(name = "email")
	private String email;

	@OneToMany(mappedBy="aluno", fetch=FetchType.EAGER, cascade = CascadeType.ALL, targetEntity = Endereco.class)
	private List<Endereco> endereco = new ArrayList<Endereco>();
      \end{lstlisting}
  \end{minipage}
  \end{center}
  
  \framebreak
  
  \item Exemplo aluno - Endereco:
  
  \begin{center}
  \begin{minipage}{0.9\textwidth}
      \lstset{
	frame=tb,
	language=Java
      }
      \begin{lstlisting}
	@Table(name="endereco")
	@Entity
	public class Endereco {
	@Id
	@SequenceGenerator(name = "idSeqEndereco", sequenceName="endereco_id_seq")
	@GeneratedValue(generator = "idSeqEndereco")
	private Integer id;

	@Column(name = "descricao")
	private String descricao;

	@JoinColumn(name = "_aluno", referencedColumnName="id")
	@ManyToOne
	private Aluno aluno;
	\end{lstlisting}
  \end{minipage}
  \end{center}
  
  \framebreak
  
  \item pode-se tratar os dados como se fossem listas de objetos;
  \item ao utilizar JSTL alterar o retorno das listas para \textit{Collection};
  \item existem m�todos de cria��o autom�tica destas classes (Netbeans);
  \item A organiza��o do modelo MVC pode se alterar de acordo com as tecnologias utilizadas;
  \begin{itemize}
    \item JSF + JPA + Facelets;
  \end{itemize}
  
\end{itemize}
\end{frame}