\section[JSP Bean]{JSP Bean}

\begin{frame}[c,allowframebreaks,fragile]
    \frametitle{JSP Bean}
    \begin{itemize}
      \item padroniza��o de um exemplo;
      \item ir� ser utilizado em todos os pr�ximos t�pios;
      \item o exemplo ser� sempre o cadastro de alunos;
      \item em determinados casos, este aluno poder� ter uma lista de endere�os;
    \end{itemize}
    \begin{center}
    \begin{minipage}{0.9\textwidth}
	\lstset{
	  frame=tb,
	  language=SQL,
	  caption=Exemplo do SQL padr�o
	}
	\begin{lstlisting}
	  CREATE TABLE alunos (
	      id integer UNIQUE NOT NULL,
	      nome text,
	      idade numeric,
	      fone text,
	      email text,
	      PRIMARY KEY (id)
	  );
	\end{lstlisting}
    \end{minipage}
    \end{center}
    
    \framebreak
    
    \begin{center}
    \begin{minipage}{0.9\textwidth}
	\lstset{
	  frame=tb,
	  language=Java,
	  caption=Exemplo de uma Bean
	}
	\begin{lstlisting}
	  package beans;

	  public class AlunoBean {
	      
	      private Integer id;
	      private Integer idade;
	      private String  nome;
	      private String  fone;
	      private String  email;
	      
	      public AlunoBean(){}
	      
	      /* gets e sets */
	      
	  }
	\end{lstlisting}
    \end{minipage}
    \end{center}
    
    \framebreak
    
    \begin{center}
    \begin{minipage}{0.9\textwidth}
	\lstset{
	  frame=tb,
	  language=HTML,
	  caption=Utiliza��o da Bean
	}
	\begin{lstlisting}
	<body>
		  <jsp:useBean id="alunoBean"
class="beans.AlunoBean" scope="request"/>
		  <jsp:setProperty property="*" name="alunoBean"/>
		  <% if(request.getParameter("nome")  != null &&
			request.getParameter("idade") != null &&
			request.getParameter("fone")  != null &&
			request.getParameter("email") != null
		      ){ 
		  %>
		  <h1>Aluno Cadastrado</h1>
		  <ul>
		      <li>Nome: <jsp:getProperty
name="alunoBean" property="nome"/></li>
		      <li>Idade: <jsp:getProperty
name="alunoBean" property="idade"/></li>
		      <li>Telefone: <jsp:getProperty
name="alunoBean" property="fone"/></li>
		      <li>Email: <jsp:getProperty
name="alunoBean" property="email"/></li>
		  </ul>
		  <% } %>
	...
	\end{lstlisting}
    \end{minipage}
    \end{center}
    
    \framebreak
    
    \begin{center}
    \begin{minipage}{0.9\textwidth}
	\lstset{
	  frame=tb,
	  language=HTML,
	  caption=Utiliza��o da Bean
	}
	\begin{lstlisting}
		  <h1>Cadastro de Aluno</h1>
		  <form action="index.jsp" method="POST">
		      <label> Nome </label>
		      <input type="text" name="nome"/> <br />
		      <label> Idade </label>
		      <input type="text" name="idade"/> <br />
		      <label> Fone </label>
		      <input type="text" name="fone"/> <br />
		      <label> Email </label>
		      <input type="text" name="email"/> <br />
		      <input type="submit" value="Enviar"/> 
		  </form>
	</body>
	\end{lstlisting}
    \end{minipage}
    \end{center}
    
\end{frame}

\begin{frame}
    \frametitle{Atividades}
      \begin{itemize}
       \item Implemente uma bean para a classe Produto a seguir:
       \begin{itemize}
	  \item id (int);
	  \item Nome (string);
	  \item Categoria (string);
	  \item Pre�o (Float);
	  \item Promo��o (Boolean);
	  \item Desconto (Float);
       \end{itemize}
       \item Os campos Promo��o e Desconto s�o opcionais;
       \item Caso o item tenha um desconto, exibir seu pre�o original, seu
pre�o com o desconto e a \% de desconto;
      \end{itemize}
\end{frame}
