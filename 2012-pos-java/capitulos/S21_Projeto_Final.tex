\section[Projeto Final]{Projeto Final}

\begin{frame}[c,allowframebreaks,fragile]
    \frametitle{Projeto Final}
    \begin{itemize}
      \item O que deve conter no projeto final?
      \begin{itemize}
	\item persist�ncia de dados com no m�nimo 03 (tr�s) e no m�ximo 05 (cinco) tabelas;
	\item utiliza��o de XHTML, CSS e Tableless para a constru��o do layout;
	\item m�todo de persist�ncia: \textbf{hibernate};
	\item prefer�ncias para o banco de dados:
	\begin{itemize}
	 \item hibernate gera o banco na primeira execuss�o;
	 \item cria��o do banco com SQL em anexo no projeto;
	 \item recupera��o do banco por seu \textit{backup};
	\end{itemize}
	\item utiliza��o de AJAX em alguma parte do sistema;
	\item utiliza��o de Sess�o e Cookie para sistema de login;
	\item utiliza��o de JSTL;
	\item utiliza��o de Beans;
	\item organiza��o do projeto: MVC;
	\item criar um arquivo texto identificando as tecnologias e onde foram utilizadas e anex�-lo ao projeto (como feito na apresenta��o do mini sistema);
	\item data de entrega: 01/01/2013;
	\item valor: 60;
      \end{itemize}
      
      \framebreak
      
      \item Instru��es para cria��o das apresenta��es dos temas:
      
      \begin{itemize}
	\item nome dos colaboradores da equipe;
	\item descri��o do projeto a ser desenvolvido;
	\item poss�veis utiliza��es das tecnologias apresentadas no curso;
	\item poss�veis classes do sistema;
	\item sugest�es da turma podem ajudar no desenvolvimento da aplica��o;
      \end{itemize}
      
    \end{itemize}
\end{frame}