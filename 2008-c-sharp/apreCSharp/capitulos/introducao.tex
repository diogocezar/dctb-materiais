\section[Introdu��o]{Introdu��o}
\begin{frame}
    \frametitle{Introdu��o}
    \begin{itemize}
        \item <1-> O que � uma plataforma de desenvolvimento?
        \item <2-> Platadorma .NET;
        \item <3-> Caracter�sticas:
            \begin{itemize}
                \item <4-> Independ�ncia de linguagem de programa��o (C\#, J\#, C++, VB);
                \item <5-> Reutiliza��o de c�digo (DLL e Outra Bibliotecas);
                \item <6-> Tempo de execu��o compartilhado (\emph{runtime} compartilhado entre as linguagens);
                \item <7-> Sistemas auto-explicativos e controle de vers�es;
                \item <8-> Simplicidade na resolu��o de problemas complexos;
                \item <9-> Multi-plataforma (independente do sistema operacional): CLR;
            \end{itemize}
    \end{itemize}
\end{frame}

\subsection[Arquitetura .NET]{Arquitetura .NET}
\begin{frame}
    \frametitle{Arquitetura .NET}
        \begin{block}<1->{CLR (\emph{Commom Language Runtime})}
            � o ambiente de execu��o das aplica��es .NET
        \end{block}
        \begin{block}<2->{CLS (\emph{Common Language Specification})}
            � um conjunto de regras que t�m como objetivo gerar uma
            s� resultante para a compila��o de qualquer uma das
            linguagens suportadas pela plataforma .NET
        \end{block}
        \begin{block}<3->{BCL (\emph{Base Classe Library})}
            � uma biblioteca de componentes de software reutiliz�veis
        \end{block}
\end{frame}
