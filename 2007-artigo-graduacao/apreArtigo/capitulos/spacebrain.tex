\section[SpaceBrain]{SpaceBrain}

\subsection[SpaceBrain]{SpaceBrain}

\begin{frame}
\frametitle{Conceitos de C�rebro e Neur�nio}
    \begin{itemize}
        \item <1-> \emph{SpaceBrain}: \emph{framework} baseado no funcionamento do c�rebro humano.
        \item <2-> neur�nios: representados por classes;
        \item <3-> cada classe possui uma fun��o e atua de forma independente;
        \item <4-> objetivo: possibilitar a invoca��o individual e n�o programada de cada neur�nio;
    \end{itemize}
\end{frame}

\subsection[Neur�nios]{Neur�nios}

\begin{frame}
\frametitle{Neur�nios}
    Com objetivo de reutiliza��o de c�digo, uma cole��o de neur�nios
    foi implementada, dentre eles est�o:

    \begin{itemize}
        \item <1-> \emph{Connection}: estabelece a comunica��o com o banco de dados;
        \item <2-> \emph{Database}: manipula��o do banco de dados;
        \item <3-> \emph{Cookie}: gerenciamento das \emph{cookies} do sistema;
        \item <4-> \emph{Session}: gerenciamento das sess�es do sistema.
    \end{itemize}
\end{frame}

%\subsection[Classe Gen�rica]{Classe Gen�rica}
%
%\begin{frame}
%\frametitle{Classe Gen�rica}
%
%    \begin{block}{Objetivo}
%    Essa � a classe respons�vel por montar o esqueleto de um objeto
%    que deva ser armazenado no banco de dados.
%    \end{block}
%
%    As suas principais funcionalidades s�o:
%
%    \begin{itemize}
%    \item \emph{uniqueKey(\$key)}: Retorna um registro �nico do banco de
%    dados com o \emph{primary key} igual a \$key passada como
%    par�metro;
%
%    \item \emph{save}: Salva os atributos do objeto atual no banco de
%    dados;
%
%    \item \emph{update}: Atualiza os dados armazenados, subistituindo-os
%    pelos atributos atuais do objeto;
%
%    \item \emph{delete}: Exclui o registro do banco de dados.
%    \end{itemize}
%
%\end{frame}
