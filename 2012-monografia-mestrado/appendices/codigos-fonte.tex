\chapter{C�digos Fonte}
\label{source-codes}

Neste ap�ndice apresenta-se exemplos de c�digos fonte utilizados para na implementa��o do sistema.

\section{Chamada do Planejador}
\label{ap-chamada-planejador}

O c�digo \ref{source_plan} mostra o arquivo que corresponde ao componente \emph{F�bricaDePlanos}. Nesse componente o m�todo \emph{doMakePlan()} � que invoca o planejador \emph{Crikey}. O sistema aguarda at� que o arquivo de sa�da seja gravado, e em seguida o utiliza para a gera��o do arquivo contendo as a��es.

\UFPRlongcode{Java}{source_plan}{Exemplo que invoca o planejador}{codes/MakePlan.java}

\section{Plano de Execu��o}
\label{ap-plano-execucao}

O c�digo \ref{execution_plab} exemplifica um plano de execu��o gerado automaticamente a partir do arquivo de a��es gerado pelo planejador.

\UFPRlongcode{Java}{execution_plab}{Exemplo de um plano de execu��o}{codes/ExecutionPlanGenerated.java}

\section{Fluxo de Trabalho de Exemplo}
\label{ap-fluxo-exemplo}

Apresenta-se um exemplo da implementa��o de um operador. O c�digo \ref{implementation} apresenta o componente que � herdado pelos operadores. O c�digo \ref{source_workflow} mostra a implementa��o de um operador para soma de matrizes.

\subsection{Implementa��o Base}
\label{ap-implementation}

\UFPRlongcode{Java}{implementation}{Implementa��o b�sica para fluxos de trabalho}{codes/Implementation.java}

\subsection{Fluxo de Trabalho para Soma de Matrizes}
\label{ap-workflow-sum}

\UFPRlongcode{Java}{source_workflow}{Fluxo de trabalho para soma de matrizes}{codes/WorkflowSumVet.java}

\section{Tradutor}
\label{ap-tradutor}

O c�digo \ref{source_translation} demonstra todo o modelo detalhado na se��o \ref{tradutor}.

\UFPRlongcode{Java}{source_translation}{C�digo que faz a tradu��o do sistema para os arquivos PDDL}{codes/WorkflowSumVet.java}