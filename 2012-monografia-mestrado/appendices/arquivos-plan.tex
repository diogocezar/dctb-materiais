\chapter{Arquivos de Planejamento}
\label{plan-files}

Neste ap�ndice apresenta-se exemplos de c�digos que representam os arquivos de dom�nio (se��o \ref{ap-dominio}), problema (se��o \ref{ap-problema}) e o plano gerado (se��o \ref{ap-plano}).

Os arquivos apresentados foram gerados automaticamente pelo tradutor que � apresentado em \ref{ap-tradutor}. O m�todo de tradu��o � detalhado em \ref{tradutor}.

\section{Dom�nio}
\label{ap-dominio}

\UFPRcode{Java}{rec_domain}{C�digo representa o dom�nio de um problema em PDDL}{codes/rec_domain.txt}

\section{Problema}
\label{ap-problema}

\UFPRcode{Java}{rec_problem}{C�digo representa um problema em PDDL}{codes/rec_problem.txt}

\section{Plano Gerado}
\label{ap-plano}

Nota-se que o plano gerado pela execu��o do plaejador \emph{CRIKEY} (se��o \ref{crikey}) apresenta a��es paralelas. A a��o $0.01$ � formada por duas instru��es, bem como a a��o $0.02$.

\UFPRcode{Java}{rec_plan}{Sa�da que representa um plano gerado pelo \emph{CRIKEY}}{codes/rec_plan.txt}
