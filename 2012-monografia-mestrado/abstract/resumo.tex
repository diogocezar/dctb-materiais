Experimentos cient�ficos produzem grande quantidade de informa��es que necessitam de processamento para uma posterior an�lise. Um cientista, que n�o � da �rea da computa��o, nem sempre possui as habilidades para desenvolver seu pr�prio ambiente de testes. Por isso a utiliza��o de executores de fluxos de trabalhos cient�ficos v�m sido largamente estudada. Uma das principais vantagens de se utilizar um processador de fluxo de trabalho cient�fico � a transpar�ncia oferecida para o cientista em rela��o a maneira com que os experimentos ser�o organizados, distribu�dos e processados. Este trabalho prop�e um modelo para cria��o de um ambiente que seja capaz de processar esses fluxos de trabalho. A �nfase est� em um escalonamento inteligente que utiliza t�cnicas para resolu��o de problemas de planejamento da �rea de intelig�ncia artificial.